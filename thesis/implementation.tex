\section{Implementation}

We implement an eBQL prototype in Rust, with \textasciitilde{}3.5k lines of Rust and
\textasciitilde{}1k lines of C. eBQL is compiled as a Rust library, allowing clients to link into
its API (defined in Section \ref{ebql-api}).

\subsection{eBQL Query Parsing}

To support parsing of eBQL's extended SQL syntax, we extend an existing Rust SQL parser,
\texttt{nom-sql}, which is based on the \texttt{nom} parser combinator framework (ref: nom-sql,
nom). We preferred \texttt{nom-sql} over other SQL parsing libraries like \texttt{sqlparser} (ref:
sqlparser), a top-down operator-precedence (TDOP) parser, due to parser generator frameworks'
general ease of extensibility (and also my personal interest).

We extend \texttt{nom-sql} with support for kernel event syntax (in standard SQL,
slashes---\texttt{/}---are not supported), and eBQL-specific operators, such as \texttt{Window},
\texttt{Histogram}, and \texttt{Quantile}, to support streaming semantics and additional analytics.

\subsection{Code Generation}
\label{impl-codegen}

Stateless operators are relatively straightforward to generate (for instance, an \texttt{Equal(a,
b)} filter becomes \texttt{if (a == b) \{ ... \}}) from the logical plan. Since the BPF stack size
is limited to 512 bytes, eBQL attempts to consolidate projects, filters, and maps and avoid storing
intermediate state on the stack.

Fully generating stateful operators like aggregations and their associated synopses can become
involved; further, the eBPF verifier requires argument types, map definitions, and function
invocations to be statically declared, preventing useful generic programming techniques in C, such
as function parameters and using \texttt{void *}. Thus, codegen is simplified by representing each
stateful operator as a composable \textit{template}; when a specific query plan is compiled into
eBPF code, the template is rendered with the actual values (in some senses, this is similar to
monomorphization; the templates represent the generic parameters, while the rendered code is the
unique instantiation).

Figure \ref{code:agg-tmpl} shows an example template for a \texttt{bpf\_for\_each\_map\_elem}
callback that fetches aggregation values. eBQL uses the Handlebars engine (ref: handlebars) to
render templates into unique instantiations.

\begin{figure}[htpb]
\begin{lstlisting}[language=C]
static __always_inline s64 __get_{{agg}}_{{field_name}}_{{query_name}}_callback(
            struct bpf_map *map,
            group_by_{{query_name}}_t *key,
            agg_t *agg,
            {{agg}}_{{field_name}}_{{query_name}}_ctx_t *ctx) {
    // Skip if aggregation value is 0; this means the value was cleared
    if (agg->val == 0) {
        return 0;
    }
    // Set agg value
    if (!ctx || !ctx->buf) {
        ERROR("Passed null context/context buffer in");
        return 1;
    }
    if (ctx->count >= ctx->buf_sz) {
        WARN("Number of aggregation results exceeds buf size; stopping...");
        return 1;
    }
    {{#each group_bys}}
    ctx->buf[ctx->count].{{field_name}} = key->{{field_name}};
    {{/each}}
    ctx->buf[ctx->count].{{agg}}_{{field_name}} = agg->val;
    ctx->count += 1;
    return 0;
}
\end{lstlisting}
\caption{A template for a callback to retrieve aggregated values using
\texttt{bpf\_for\_each\_map\_elem}.}
\label{code:agg-tmpl}
\end{figure}

Using these templates, the internal stateful operator implementations can be exposed via a single
helper function; thus, in the actual eBPF program code, only that helper function needs to be
invoked, greatly simplifying the codegen into a sequence of operators. A generated eBPF program for
the program in Figure \ref{code:ebql-ex} might then look like this:

\begin{lstlisting}[language=C]
SEC("tp/syscalls/sys_enter_pread64")
u32 pread_query(struct trace_event_raw_sys_enter* ctx) {
    u64 pid = PID();
    if (pid == 1041370) {
        return 1;
    }
    u64 time = TIME();
    u64 fd = ctx->args[0];
    u64 cpu = CPU();
    u64 count = ctx->args[2];
    bool tumble = window_add(time);
    if (tumble) {
        // window tumbling and emitting to user-space logic...
    }
    insert_count__pread_query({fd, cpu}, 1);
    insert_max_count_pread_query({fd, cpu}, count);
    insert_avg_count_pread_query({fd, cpu}, count);
    return 0;
}
\end{lstlisting}

\subsection{Prototype Limitations}

Being a prototype, eBQL has many limitations.

eBQL's query plan analysis is limited, restricting the complexity of generated queries; in
particular, nested aggregations, histograms/quantiles, nested selects, post-aggregation processing,
and joins are not currently supported. Part of this is by design: due to its context of executing in
resource-constrained, performance-sensitive environments, in-kernel queries should incur minimal
overhead and avoid unpredictable runtimes that could spike tail latencies (for instance, a large
binary join every second could skyrocket p99s). Part of it is from verifier restrictions: without
support for dynamic memory or unbounded loops, programs must pre-allocate potentially wasteful
amounts of memory, or are not feasibly implementable (i.e. joins and arbitrary windows).

Part of its limitations is simply due to the code generation step; the method is rather simple,
and does not handle more complicated ASTs yet. Thus, while templated operators enable ease of code
generation, complicated control flow amounts to manual checks and implementation. In the future, it
would be worth exploring potentially harnessing LLVM IR directly (as \texttt{bpftrace} does) to
generate eBPF code, or implementing a more sophisticated compiler.

eBQL currently only supports queries that are feasibly implementable in kernel space; queries that
require complex post-processing in user-space are not supported.


% references:
% \begin{itemize}
    % \item nom-sql, nom
    % \item sqlparser
    % \item handlebars
% \end{itemize}
